@article{Jakkala2021,
    author = {Jakkala, Sai Guruprasad and Vengadesan, S.},
    title = "{Study on the Applicability of URANS, Large Eddy Simulations, and Hybrid Large Eddy Simulations/RANS Models for Prediction of Hydrodynamics of Cyclone Separator }",
    journal = {Journal of Fluids Engineering},
    volume = {144},
    number = {3},
    year = {2021},
    month = {10},
    abstract = "{Cyclone separators are an integral part of many industrial processes. A good understanding of the flow features is paramount to efficiently use them. The turbulent fluid flow characteristics are modeled using unsteady Reynolds-averaged Navier–Stokes (URANS), large eddy simulations (LES), and hybrid LES/Reynolds–averaged Navier–Stokes (RANS) turbulent models. The hybrid LES/RANS approaches, namely, detached eddy simulation (DES), delayed detached eddy simulation (DDES), and improved delayed detached eddy simulation (IDDES) based on the k−ω SST RANS approaches are explored. The study is carried out for three different inlet velocities (v = 8, 16.1, and 32 m/s). The results from hybrid LES/RANS models are shown to be in good agreement with the experimental data available in the literature. Reduction in computational time and mesh size are the two main benefits of using hybrid LES/RANS models over the traditional LES methods. The Reynolds stresses are observed in order to understand the redistribution of turbulent energy in the flow field. The velocity profiles and vorticity quantities are explored to obtain a better understanding of the behavior of fluid flow in cyclone separators.}",
    issn = {0098-2202},
    doi = {10.1115/1.4052050},
    url = {https://doi.org/10.1115/1.4052050},
    note = {031501},
    eprint = {https://asmedigitalcollection.asme.org/fluidsengineering/article-pdf/144/3/031501/6772463/fe\_144\_03\_031501.pdf},
}
